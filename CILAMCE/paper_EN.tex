% Options for packages loaded elsewhere
\PassOptionsToPackage{unicode}{hyperref}
\PassOptionsToPackage{hyphens}{url}
%

%% --------------------------------------------------------------------------
% LaTeX template for the XL CILAMCE.
%
% This latex document tries to copy the Microsoft Word template.
% --------------------------------------------------------------------------
\documentclass[a4paper,11pt]{book}

% PACKAGES USED - packages that need to be previously installed on your computer
\usepackage[lmargin=2.5cm, rmargin=2.5cm, tmargin=2.5cm, bmargin=2.5cm ]{geometry}
\usepackage{graphicx}
\usepackage{times}
\usepackage{indentfirst}
\usepackage{fancyhdr}
\usepackage{titlesec}
\usepackage[english]{babel}
\usepackage{parskip} 
\usepackage{setspace}

%%%%%%%%%%%%%%%%%%%%%%%%%%%%%%%%%%%%%%%%%%%%%%%%%%%%%%%%%%%%%%%%%
%%%%%%%%%%%%%%%%%%%%%%%%%%%%%%%%%%%%%%%%%%%%%%%%%%%%%%%%%%%%%%%%%
%%% My Additional Packages
%%%%%%%%%%%%%%%%%%%%%%%%%%%%%%%%%%%%%%%%%%%%%%%%%%%%%%%%%%%%%%%%%
\usepackage[utf8]{inputenc}
\usepackage{hyperref}
%\usepackage{cleveref}
% \usepackage{./pkg-crefNames}
\usepackage[labelsep=period]{caption}

% % %BibTeX compatible with the CILAMCE format
% \usepackage[numbers,sort&compress]{natbib}
% \makeatletter
% \renewcommand\bibsection
% {
%   \section*{References}
% }
% \makeatother
% \bibliographystyle{./bib-cilamce}
% \setlength{\bibhang}{0pt}

%%%%%%%%%%%%%%%%%%%%%%%%%%%%%%%%%%%%%%%%%%%%%%%%%%%%%%%%%%%%%%%%%
%%%%%%%%%%%%%%%%%%%%%%%%%%%%%%%%%%%%%%%%%%%%%%%%%%%%%%%%%%%%%%%%%

% CONFIGURATION
\renewcommand*\arraystretch{1.5}
\renewcommand*\thesection{\arabic{section}}
%\hyphenpenalty=10000 % You can uncomment this to avoid hyphenization
\titleformat*{\section}{\large\bfseries}
\titleformat*{\subsection}{\bfseries}
\titlespacing\section{0pt}{18pt plus 2pt minus 2pt}{18pt plus 2pt minus 2pt}
\titlespacing\subsection{0pt}{13pt plus 0pt minus 2pt}{13pt plus 0pt minus 2pt}
\setlength{\parskip}{0pt} % Spacing between paragraphs
\setlength{\parindent}{0.75cm} % Paragraph identation
\setlength\abovecaptionskip{6pt}

% --------------------------------------------------------------------------
% DO NOT EDIT - SPECIAL HEADINGS OF XLI CILAMCE
% --------------------------------------------------------------------------
\fancypagestyle{first}
{
\fancyhf{}
\fancyfoot[RO]{\footnotesize \textit{CILAMCE 2020 \\
Proceedings of the XLI Ibero-Latin-American Congress on Computational Methods in Engineering, ABMEC.\\
Foz do Iguaçu/PR, Brazil, November 16-19, 2020}}
\renewcommand{\headrulewidth}{.0pt}
\renewcommand{\footrulewidth}{.5pt}
}

\pagestyle{fancy}
\fancyhf{}
\fancyhead[RO,LE]{}

\fancyfoot[RO]{\footnotesize \textit{CILAMCE 2020 \\
Proceedings of the XLI Ibero-Latin-American Congress on Computational Methods in Engineering, ABMEC.\\
Foz do Iguaçu/PR, Brazil, November 16-19, 2020}}

\fancyfoot[LE]{\footnotesize \textit{CILAMCE 2020 \\
Proceedings of the XLI Ibero-Latin-American Congress on Computational Methods in Engineering, ABMEC.\\
Foz do Iguaçu/PR, Brazil, November 16-19, 2020}}

\renewcommand{\headrulewidth}{.5pt}
\renewcommand{\footrulewidth}{.5pt}

% --------------------------------------------------------------------------
% PLEASE, EDIT THIS!
\fancyhead[LE]{\footnotesize \textit{Concrete compressive strength prediction with machine learning}}
\fancyhead[RO]{\footnotesize \textit{P. Moreira, V. da Silva}}
% --------------------------------------------------------------------------

%%%%%%%%%%%%%%%%%%%%%%%%%%%%%%%%%%%%%%%%%%%%%%%%%%%%%%%%%%%%%%%%%
%%%%%%%%%%%%%%%%%%%%%%%%%%%%%%%%%%%%%%%%%%%%%%%%%%%%%%%%%%%%%%%%%

\usepackage{lmodern}
\usepackage{amssymb,amsmath}
\usepackage{ifxetex,ifluatex}
\ifnum 0\ifxetex 1\fi\ifluatex 1\fi=0 % if pdftex
  \usepackage[T1]{fontenc}
  \usepackage[utf8]{inputenc}
  \usepackage{textcomp} % provide euro and other symbols
\else % if luatex or xetex
  \usepackage{unicode-math}
  \defaultfontfeatures{Scale=MatchLowercase}
  \defaultfontfeatures[\rmfamily]{Ligatures=TeX,Scale=1}
  \setmainfont[]{Times New Roman}
\fi
% Use upquote if available, for straight quotes in verbatim environments
\IfFileExists{upquote.sty}{\usepackage{upquote}}{}
\IfFileExists{microtype.sty}{% use microtype if available
  \usepackage[]{microtype}
  \UseMicrotypeSet[protrusion]{basicmath} % disable protrusion for tt fonts
}{}
\usepackage{xcolor}
\IfFileExists{xurl.sty}{\usepackage{xurl}}{} % add URL line breaks if available
\IfFileExists{bookmark.sty}{\usepackage{bookmark}}{\usepackage{hyperref}}
\hypersetup{
  pdftitle={INSTRUCTIONS FOR PREPARATION AND SUBMISSION OF FULL-PAPERS FOR PUBLICATION IN THE PROCEEDINGS OF XL CILAMCE},
  hidelinks,
  pdfcreator={LaTeX via pandoc}}
\urlstyle{same} % disable monospaced font for URLs
\usepackage{color}
\usepackage{fancyvrb}
\newcommand{\VerbBar}{|}
\newcommand{\VERB}{\Verb[commandchars=\\\{\}]}
\DefineVerbatimEnvironment{Highlighting}{Verbatim}{commandchars=\\\{\}}
% Add ',fontsize=\small' for more characters per line
\usepackage{framed}
\definecolor{shadecolor}{RGB}{248,248,248}
\newenvironment{Shaded}{\begin{snugshade}}{\end{snugshade}}
\newcommand{\AlertTok}[1]{\textcolor[rgb]{0.94,0.16,0.16}{#1}}
\newcommand{\AnnotationTok}[1]{\textcolor[rgb]{0.56,0.35,0.01}{\textbf{\textit{#1}}}}
\newcommand{\AttributeTok}[1]{\textcolor[rgb]{0.77,0.63,0.00}{#1}}
\newcommand{\BaseNTok}[1]{\textcolor[rgb]{0.00,0.00,0.81}{#1}}
\newcommand{\BuiltInTok}[1]{#1}
\newcommand{\CharTok}[1]{\textcolor[rgb]{0.31,0.60,0.02}{#1}}
\newcommand{\CommentTok}[1]{\textcolor[rgb]{0.56,0.35,0.01}{\textit{#1}}}
\newcommand{\CommentVarTok}[1]{\textcolor[rgb]{0.56,0.35,0.01}{\textbf{\textit{#1}}}}
\newcommand{\ConstantTok}[1]{\textcolor[rgb]{0.00,0.00,0.00}{#1}}
\newcommand{\ControlFlowTok}[1]{\textcolor[rgb]{0.13,0.29,0.53}{\textbf{#1}}}
\newcommand{\DataTypeTok}[1]{\textcolor[rgb]{0.13,0.29,0.53}{#1}}
\newcommand{\DecValTok}[1]{\textcolor[rgb]{0.00,0.00,0.81}{#1}}
\newcommand{\DocumentationTok}[1]{\textcolor[rgb]{0.56,0.35,0.01}{\textbf{\textit{#1}}}}
\newcommand{\ErrorTok}[1]{\textcolor[rgb]{0.64,0.00,0.00}{\textbf{#1}}}
\newcommand{\ExtensionTok}[1]{#1}
\newcommand{\FloatTok}[1]{\textcolor[rgb]{0.00,0.00,0.81}{#1}}
\newcommand{\FunctionTok}[1]{\textcolor[rgb]{0.00,0.00,0.00}{#1}}
\newcommand{\ImportTok}[1]{#1}
\newcommand{\InformationTok}[1]{\textcolor[rgb]{0.56,0.35,0.01}{\textbf{\textit{#1}}}}
\newcommand{\KeywordTok}[1]{\textcolor[rgb]{0.13,0.29,0.53}{\textbf{#1}}}
\newcommand{\NormalTok}[1]{#1}
\newcommand{\OperatorTok}[1]{\textcolor[rgb]{0.81,0.36,0.00}{\textbf{#1}}}
\newcommand{\OtherTok}[1]{\textcolor[rgb]{0.56,0.35,0.01}{#1}}
\newcommand{\PreprocessorTok}[1]{\textcolor[rgb]{0.56,0.35,0.01}{\textit{#1}}}
\newcommand{\RegionMarkerTok}[1]{#1}
\newcommand{\SpecialCharTok}[1]{\textcolor[rgb]{0.00,0.00,0.00}{#1}}
\newcommand{\SpecialStringTok}[1]{\textcolor[rgb]{0.31,0.60,0.02}{#1}}
\newcommand{\StringTok}[1]{\textcolor[rgb]{0.31,0.60,0.02}{#1}}
\newcommand{\VariableTok}[1]{\textcolor[rgb]{0.00,0.00,0.00}{#1}}
\newcommand{\VerbatimStringTok}[1]{\textcolor[rgb]{0.31,0.60,0.02}{#1}}
\newcommand{\WarningTok}[1]{\textcolor[rgb]{0.56,0.35,0.01}{\textbf{\textit{#1}}}}
\setlength{\emergencystretch}{3em} % prevent overfull lines
\providecommand{\tightlist}{%
  \setlength{\itemsep}{0pt}\setlength{\parskip}{0pt}}
\setcounter{secnumdepth}{5}
\usepackage{booktabs}
\usepackage{longtable}
\usepackage{array}
\usepackage{multirow}
\usepackage{wrapfig}
\usepackage{float}
\usepackage{colortbl}
\usepackage{pdflscape}
\usepackage{tabu}
\usepackage{threeparttable}
\usepackage{threeparttablex}
\usepackage[normalem]{ulem}
\usepackage{makecell}
\usepackage{xcolor}
% \usepackage[]{natbib}
\usepackage[numbers,sort&compress,sectionbib]{natbib}
\bibliographystyle{./bib-cilamce}

\title{INSTRUCTIONS FOR PREPARATION AND SUBMISSION OF FULL-PAPERS FOR
PUBLICATION IN THE PROCEEDINGS OF XL CILAMCE}
\author{}
\date{}

\begin{document}\thispagestyle{first}

\begin{figure}[ht!]
\vspace{-50pt}
\flushright
\includegraphics[width=5.5cm]{logoCILAMCE2020.png}
%scale=0.25
\end{figure}

% \maketitle
% 
% manually adding title
\noindent
\textbf{\large
INSTRUCTIONS FOR PREPARATION AND SUBMISSION OF FULL-PAPERS FOR
PUBLICATION IN THE PROCEEDINGS OF XL CILAMCE}
\vspace{18pt}

\noindent  \textbf{Pedro B. A. Moreira}

\noindent  \textit{pedrobermoreira@gmail.com}

\noindent \textit{Student, Dept. of Engineering, University Veiga de Almeida}

\noindent \textit{Address, Zip-Code, State/Province, Country}

\noindent  \textbf{Victor M. da Silva}

\noindent  \textit{somebody2@somewhere.com}

\noindent \textit{Assistant Professor, Dept. of Engineering, IBMEC/RJ}

\noindent \textit{Address, Zip-Code, State/Province, Country}

\vspace{18pt}

\noindent \textbf{Abstract.}
Compressive strength is the main characteristic of concrete. The correct prediction of this parameter results in cost and time reduction. This work built predictive models for 6 different ages of concrete samples (3, 7, 14, 28, 56, and 100 days). Was used a dataset with 9 variables: compressive strength, age, and 7 ingredients (water, cement, fine aggregate, coarse aggregate, fly ash, blast furnace slag, and superplasticizers). Another 6 variables were added to represent the proportions of the main ingredients in each sample (water/cement, fine aggregate/cement, coarse aggregate/cement, fine aggregate/coarse aggregate, water/coarse aggregate, and water/fine aggregate). The predictive models were developed in R language, using the caret package with the Parallel Random Forest algorithm and repeated cross-validation technique to optimize the parameters. The results were satisfactory and compatible with other studies using the same data set. The most important model, 28 days old, obtained RMSE of 4.717. The 3-day model obtained the best result, RMSE of 3.310. The worst result was the 56-day model, with RMSE of 5.939. The work showed that the compressive strength of concrete can be predicted. The choice of creating a model for each age, instead of using age as a predictor, allowed to get compatible results with the available data at each age. It was a promising alternative since good results were achieved by training with just one algorithm. This work facilitates exploration and new efforts to predict the compressive strength of concrete, it can be replicated using different algorithms or the combination of several.

\vspace{18pt} % <- keep this vertical space!

\noindent \textbf{Keywords:} Concrete, Compressive Strength, Machine Learning, Prediction

\pagenumbering{gobble}

\clearpage



\renewcommand{\bibname}{References}

\setlength{\abovedisplayskip}{18pt}
\setlength{\belowdisplayskip}{18pt}
\setlength{\abovedisplayshortskip}{18pt}
\setlength{\belowdisplayshortskip}{18pt}

\renewcommand{\arraystretch}{1.2}

\captionsetup[table]{skip=12pt}

\setlength\intextsep{13pt}

\hypertarget{introduction}{%
\section{Introduction}\label{introduction}}

Compressive strength is the main characteristic of concrete, measured by
tests of international standards that consist of the breaking of
specimens. Measurement at 28 days is mandatory and represents the grade
of the concrete. Knowing in advance what the result will be obtained for
a given age, based on the proportions of its ingredients, is of great
interest to concrete manufacturers, construction companies, and civil
engineers.

This compressive strength is a nonlinear function of its ingredients and
age, making it difficult to establish an analytical formula, although
some formulas have already been proposed. \citet{Hasan2011} proposed a
mathematical model to predict from the results of tests of 7 and 14
days, and \citet{Kabir2012} from 7 days. However, machine learning
techniques can be used to model this characteristic from real sample
data, using only the ingredients.

\begin{Shaded}
\begin{Highlighting}[]
\CommentTok{# Aqui vale incluir um parágrafo para explicar o que é machine learning,}
\CommentTok{# e por que é possível obter a resistência a partir apenas dos ingredientes.}
\end{Highlighting}
\end{Shaded}

Many previous studies use the same dataset used by \citet{Yeh1998} to
predict the compressive strength of concrete. \citet{Alshamiri2020} got
good results with the regularized extreme learning machine (RELM)
technique, and \citet{Hameed2020} got even better results with the
Artificial Neural Networks and cross-validation technique. This set of
samples is so well known that there are many pages on the internet of
unpublished studies that use it and have good results, such as
\citet{Abban2016}, \citet{Raj2018}, \citet{Modukuru2020} and
\citet{Pierobon2018}. At the end of the work, the results found are
compared to the works cited here.

Unlike previous studies with this dataset, this work does data
preparation differently. The age of the concrete is the most unique
feature that contributes to its compressive strength. For this reason,
age is treated separately in the machine learning models, creating
models for each age group.

\hypertarget{materials-and-methods}{%
\section{Materials and Methods}\label{materials-and-methods}}

\hypertarget{materials}{%
\subsection{Materials}\label{materials}}

\begin{Shaded}
\begin{Highlighting}[]
\CommentTok{# Editar, porque não vai ter mais os apêndices}
\end{Highlighting}
\end{Shaded}

The methodology was carried out using RStudio software \citep{RStudio},
an integrated virtual environment for code development in \emph{R}
\citep{RCore}. Throughout the process, all code executed was documented
in the same order as its execution in Appendix 3, and reference was
always made to codes throughout the text. All relevant information
related to the operating system and installed packages has been
presented in Appendix 1. In addition, an online and open-source
repository was created in \emph{Github}, housing all the code used to
generate this work, the link was made available in{[}Appendix 2.

\hypertarget{reproducibility}{%
\subsection{Reproducibility}\label{reproducibility}}

\begin{Shaded}
\begin{Highlighting}[]
\CommentTok{# Editar, porque não vai ter mais os apêndices}
\CommentTok{# Definir o que é seed}
\end{Highlighting}
\end{Shaded}

In order to guarantee reproducibility, whenever there was some code that
could use probabilistic operations, a \emph{seed} was defined before its
execution, ensuring that when run on another machine, with the same
version of \emph{R} and the same \emph{seed}, get the same result. The
\emph{seeds} can be checked throughout Appendix 3 or directly on
\emph{Github}.

\hypertarget{obtaining-the-data}{%
\subsection{Obtaining the data}\label{obtaining-the-data}}

The data was downloaded from the University of California Irvine website
\citep{downloadData}. In total there are 1030 samples with 9 columns.
The samples were renamed and an id column was added to facilitate data
manipulation. The columns were reordered to put the new id column in the
first position. The first samples can be viewed in the Table
\ref{tab:first-samples}.

\begin{table}[H]

\caption{\label{tab:first-samples}First 6 samples}
\centering
\resizebox{\linewidth}{!}{
\begin{tabular}[t]{cccccccccc}
\toprule
\multicolumn{1}{c}{ID} & \multicolumn{1}{c}{Cement} & \multicolumn{1}{c}{B.F.S.} & \multicolumn{1}{c}{Fly ash} & \multicolumn{1}{c}{Water} & \multicolumn{1}{c}{Superp.} & \multicolumn{1}{c}{C.Aggregate} & \multicolumn{1}{c}{F.Aggregate} & \multicolumn{1}{c}{Day} & \multicolumn{1}{c}{Comp.Str.} \\
 & $kg/m^3$ & $kg/m^3$ & $kg/m^3$ & $kg/m^3$ & $kg/m^3$ & $kg/m^3$ & $kg/m^3$ &  & $MPa$\\
\midrule
1 & 540.0 & 0.0 & 0 & 162 & 2.5 & 1040.0 & 676.0 & 28 & 79.99\\
2 & 540.0 & 0.0 & 0 & 162 & 2.5 & 1055.0 & 676.0 & 28 & 61.89\\
3 & 332.5 & 142.5 & 0 & 228 & 0.0 & 932.0 & 594.0 & 270 & 40.27\\
4 & 332.5 & 142.5 & 0 & 228 & 0.0 & 932.0 & 594.0 & 365 & 41.05\\
5 & 198.6 & 132.4 & 0 & 192 & 0.0 & 978.4 & 825.5 & 360 & 44.30\\
6 & 266.0 & 114.0 & 0 & 228 & 0.0 & 932.0 & 670.0 & 90 & 47.03\\
\bottomrule
\end{tabular}}
\end{table}

\hypertarget{data-preparation}{%
\subsection{Data preparation}\label{data-preparation}}

The preparation of the data consisted of transforming the sample set in
order to maintain only relevant data for the subsequent studies. Data
that were considered irrelevant or that had the potential to add
undesirable noise to the analysis were removed. In addition, the
relevant data has been transformed to better fit the studies in the next
steps.

\hypertarget{initial-data-cleaning}{%
\subsubsection{Initial data cleaning}\label{initial-data-cleaning}}

Initially, there were 25 duplicate samples that were removed, resulting
in a new total of 1005 samples.

The data show the variables in the columns and samples in the rows.
However it was found that some samples are identical in proportions of
ingredients, changing only the value of age and compressive strength,
for example, samples 653, 654, 678 and 681, shown in the Table
\ref{tab:similar-samples}.

\begin{table}[H]

\caption{\label{tab:similar-samples}Samples with same composition}
\centering
\resizebox{\linewidth}{!}{
\begin{tabular}[t]{cccccccccc}
\toprule
\multicolumn{1}{c}{ID} & \multicolumn{1}{c}{Cement} & \multicolumn{1}{c}{B.F.S.} & \multicolumn{1}{c}{Fly ash} & \multicolumn{1}{c}{Water} & \multicolumn{1}{c}{Superp.} & \multicolumn{1}{c}{C.Aggregate} & \multicolumn{1}{c}{F.Aggregate} & \multicolumn{1}{c}{Day} & \multicolumn{1}{c}{Comp.Str.} \\
 & $kg/m^3$ & $kg/m^3$ & $kg/m^3$ & $kg/m^3$ & $kg/m^3$ & $kg/m^3$ & $kg/m^3$ &  & $MPa$\\
\midrule
653 & 102 & 153 & 0 & 192 & 0 & 887 & 942 & 3 & 4.57\\
\addlinespace
678 & 102 & 153 & 0 & 192 & 0 & 887 & 942 & 7 & 7.68\\
\addlinespace
681 & 102 & 153 & 0 & 192 & 0 & 887 & 942 & 28 & 17.28\\
\addlinespace
654 & 102 & 153 & 0 & 192 & 0 & 887 & 942 & 90 & 25.46\\
\bottomrule
\end{tabular}}
\end{table}

In addition, there are also samples with the same values and proportions
of ingredients, but with different compressive strength, probably due to
differences in the building process. This is the case, for example, of
samples 472, 473 and 474, shown in the table
\ref{tab:similar-samples-2}.

\begin{center}\rule{0.5\linewidth}{0.5pt}\end{center}

\hypertarget{lorem-ipsum-sub}{%
\subsection{Lorem ipsum Sub}\label{lorem-ipsum-sub}}

Lorem ipsum dolor sit amet, consectetur adipiscing elit, sed do eiusmod
tempor incididunt ut labore et dolore magna aliqua. Ut enim ad minim
veniam, quis nostrud exercitation ullamco laboris nisi ut aliquip ex ea
commodo consequat. Duis aute irure dolor in reprehenderit in voluptate
velit esse cillum dolore eu fugiat nulla pariatur.

\hypertarget{lorem-ipsum-sub-2}{%
\subsection{Lorem ipsum Sub 2}\label{lorem-ipsum-sub-2}}

Lorem ipsum dolor sit amet, consectetur adipiscing elit, sed do eiusmod
tempor incididunt ut labore et dolore magna aliqua. Ut enim ad minim
veniam, quis nostrud exercitation ullamco laboris nisi ut aliquip ex ea
commodo consequat. Duis aute irure dolor in reprehenderit in voluptate
velit esse cillum dolore eu fugiat nulla pariatur.

Lorem ipsum dolor sit amet, consectetur adipiscing elit, sed do eiusmod
tempor incididunt ut labore et dolore magna aliqua. Ut enim ad minim
veniam, quis nostrud exercitation ullamco laboris nisi ut aliquip ex ea
commodo consequat. Duis aute irure dolor in reprehenderit in voluptate
velit esse cillum dolore eu fugiat nulla pariatur.

Lorem ipsum dolor sit amet, consectetur adipiscing elit, sed do eiusmod
tempor incididunt ut labore et dolore magna aliqua. Ut enim ad minim
veniam, quis nostrud exercitation ullamco laboris nisi ut aliquip ex ea
commodo consequat. Duis aute irure dolor in reprehenderit in voluptate
velit esse cillum dolore eu fugiat nulla pariatur.

\hypertarget{equations-symbols-and-units}{%
\subsection{Equations, symbols and
units}\label{equations-symbols-and-units}}

Lorem ipsum dolor sit amet, consectetur adipiscing elit, sed do eiusmod
tempor incididunt ut labore et dolore magna aliqua. Ut enim ad minim
veniam, quis nostrud exercitation ullamco laboris nisi ut aliquip ex ea
commodo consequat. Duis aute irure dolor in reprehenderit in voluptate
velit esse cillum dolore eu fugiat nulla pariatur.

\begin{center}
\begin{equation}
q_r = -4pr^2k\frac{dT}{dr}.
\label{Eq1}
\end{equation}
\end{center}

Lorem ipsum dolor sit amet, consectetur adipiscing elit, sed do eiusmod
tempor incididunt ut labore et dolore magna aliqua. Ut enim ad minim
veniam, quis nostrud exercitation ullamco laboris nisi ut aliquip ex ea
commodo consequat. Duis aute irure dolor in reprehenderit in voluptate
velit esse cillum dolore eu fugiat nulla pariatur.

\vspace{13pt}

Cite as Eq. \ref{Eq1}.

\hypertarget{figures-and-tables}{%
\subsection{Figures and tables}\label{figures-and-tables}}

Lorem ipsum dolor sit amet, consectetur adipiscing elit, sed do eiusmod
tempor incididunt ut labore et dolore magna aliqua. Ut enim ad minim
veniam, quis nostrud exercitation ullamco laboris nisi ut aliquip ex ea
commodo consequat. Duis aute irure dolor in reprehenderit in voluptate
velit esse cillum dolore eu fugiat nulla pariatur.

\begin{table}[H]

\caption{\label{tab:table-example}Coefficients in constitutive relations}
\centering
\begin{tabular}[t]{ccc}
\toprule
Constitutive relation & Nomenclature & Value\\
\midrule
Turbulent tensor & C & 0.09\\
Turbulent tensor & C & 0.69\\
Lateral lift & C & 0.08\\
Virtual mass & C & 0.80\\
\bottomrule
\end{tabular}
\end{table}

Cite as Figure \ref{fig:plot-example} and Table \ref{tab:table-example}.

\begin{figure}[!ht]

{\centering \includegraphics{paper_EN_files/figure-latex/plot-example-1} 

}

\caption{Pressure variation along the nozzle: experimental data}\label{fig:plot-example}
\end{figure}

Lorem ipsum dolor sit amet, consectetur adipiscing elit, sed do eiusmod
tempor incididunt ut labore et dolore magna aliqua. Ut enim ad minim
veniam, quis nostrud exercitation ullamco laboris nisi ut aliquip ex ea
commodo consequat. Duis aute irure dolor in reprehenderit in voluptate
velit esse cillum dolore eu fugiat nulla pariatur.

\hypertarget{tes}{%
\section{tes}\label{tes}}

Example of reference: "\citet{Yeh1998} proposed\ldots{}
\citet{Hameed2020}. \citet{Alshamiri2020}, ipsum dolor sit amet,
consectetur adipiscing elit, sed do eiusmod tempor incididunt ut labore
et dolore magna aliqua. Ut enim ad minim veniam, quis nostrud
exercitation ullamco laboris nisi ut aliquip ex ea commodo consequat.
Duis aute irure dolor in reprehenderit in voluptate velit esse cillum
dolore eu fugiat nulla pariatur. Lorem ipsum dolor sit amet, consectetur
adipiscing elit, sed do eiusmod tempor incididunt ut labore et dolore
magna aliqua. Ut enim ad minim veniam, quis nostrud exercitation ullamco
laboris nisi ut aliquip ex ea commodo consequat. Duis aute irure dolor
in reprehenderit in voluptate velit esse cillum dolore eu fugiat nulla
pariatur.

% 
  \bibliography{references.bib}

% 
\end{document}
